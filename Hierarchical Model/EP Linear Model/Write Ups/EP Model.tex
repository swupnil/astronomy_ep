\documentclass[12pt]{article}
\usepackage{amssymb,amsmath,graphicx,mathtools}
\usepackage{listings}
\usepackage[margin=0.75in]{geometry}
\parindent 16 pt
\usepackage{fancyhdr}
\pagestyle{fancy}
\fancyhead[R]{Swupnil Sahai}
\fancyhead[L]{Expectation Propogation Model}
\DeclarePairedDelimiter\ceil{\lceil}{\rceil}
\DeclarePairedDelimiter\floor{\lfloor}{\rfloor}

\lstset{
    language=R,
    basicstyle=\scriptsize\ttfamily,
    stepnumber=1,
    numbersep=5pt,
    showspaces=false,
    showstringspaces=false,
    showtabs=false,
    frame=single,
    tabsize=2,
    captionpos=b,
    breaklines=true,
    breakatwhitespace=false,
    escapeinside={},
    keywordstyle={},
    morekeywords={}
    }

\begin{document}

% CUSTOM SHORTCUTS

\def\ci{\perp\!\!\!\perp}
\def\ex{\mathbb{E}}
\def\prob{\mathbb{P}}
\def\ind{\mathbb{I}}
\def\grad{\triangledown}
\def\bigo{\mathcal{O}}

\section*{EP Set Up and Algorithm}
We wish to split the sky map into $J=360$ bins, regressing FUV ($y_{ij}$) on i100 ($x_{ij}$) within each bin $j$. To start we'll fit a constant slope $\beta$ but a varying intercept $a_j$:
$$ y_{ij} \sim N(a_j + b x_{ij}, \sigma^2) \hspace{30 pt} \alpha_j \sim N(\theta, \tau^2)$$

\noindent Letting $K=6$ and using the notation $\phi = (b,\log \sigma,\theta,\log \tau)$, we have that the posterior can be written as:
$$ p( {\bf a}, b, \theta,\log \tau, \log \sigma | y) = f(\phi) \prod_{k=1}^6 f_k(\phi, {\bf a}_{(k)}) = f(\phi) \prod_{k=1}^6 f_k({\bf a}_{(k)} | \phi) f_k (\phi)$$
$$ = p(\phi) \prod_{k=1}^6 p({\bf a}_{(k)} | \theta,\log \tau) p(y | {\bf a}_{(k)}, b,\log \sigma)$$

\noindent Then, utilizing a flat prior over $\phi$, we get that for our approximation, $g_0(\phi)$ is a constant, so that our approximation then becomes:
$$ g(\phi) = \prod_{k=1}^6 g_k(\phi) = \prod_{k=1}^6 N(\phi| \mu_k,\Sigma_k)$$

\noindent Our initial approximation then becomes, for all $k$, $\mu_k = {\bf 0}$ and $\Sigma_k = 10^2 {\bf I} / 6$ (note that the dimension of $\mu_k$ is equal to the dimension of $\phi$, which is 4). The algorithm then proceeds as follows:\\

\indent (a) Compute the cavity distribution $g_{-k}(\phi) = \frac{g(\phi)}{g_k(\phi)} = \prod_{k' \neq k} N(\phi | \mu_{k'} , \Sigma_{k'}) = N( \mu_{-k},\Sigma_{-k})$ where:
$$ \Sigma_{-k}^{-1} = \sum_{k' \neq k} \Sigma_{k'}^{-1}$$
$$ \Sigma_{-k}^{-1}\mu_{-k} = \sum_{k' \neq k} \Sigma_{k'}^{-1} \mu_{k'}$$

\indent (b.i) Approximate\footnote{This could possibly be done in Stan, though unsure how to approximate 64x64-dimensional covariance matrix.} the tilted distribution $g_{\backslash k} ({\bf a}_{(k)},\phi) \approx N(\mu_{\backslash k}^*,\Sigma_{\backslash k}^*)$ where:
$$ g_{\backslash k} ({\bf a}_{(k)}) = g_{-k}(\phi) p({\bf a}_{(k)},y_{(k)}|\phi) = N(\phi | \mu_{-k}, \Sigma_{-k}) \prod_{j\in (k)} N(a_j | \theta, \tau) p(y_j | a_j, b, \sigma^2)$$

\indent (b.ii) Approximate\footnote{This could possibly be done in Stan, though unsure how to approximate 4x4-dimensional covariance matrix.} the marginal tilted distribution $g_{\backslash k} (\phi) = \int g_{\backslash k} ({\bf a}_{(k)},\phi) d{\bf a_{(k)}} \approx N(\mu_{\backslash k},\Sigma_{\backslash k})$.\\

\indent (c) Update\footnote{This may result in non-positive-definite inverse matrices.} the site distribution $g_k^{new} = \frac{g_{\backslash k}(\phi)}{g_{-k}(\phi)} = N(\mu_k^{new},\Sigma_k^{new})$ where:
$$ (\Sigma_k^{new})^{-1} = (\Sigma_{\backslash k})^{-1} - \Sigma_{-k}^{-1}$$
$$ (\Sigma_k^{new})^{-1} \mu_k^{new} = (\Sigma_{\backslash k})^{-1}\mu_{\backslash k} - \Sigma_{-k}^{-1}\mu_{-k}$$

\indent (d) Update $g(\phi)$ in serial or parallel.

\section*{Complications}
\noindent (b.i) This could be run in Stan as mentioned in the paper. Approximating $\mu_{\backslash k}^*$ would be easy as we would just take the means of each parameters' simulations. However, I'm unsure if we could use sample covariances of the simulations to approximate $\Sigma_{\backslash k}^*$. \\

\noindent (b.ii) Upon running the MCMC simulation in (b.i), we could once again simply approximate $\mu_{\backslash k}$ by taking the means of only the four $\phi$ parameters' simulations. (This actually obviates the need to calculate $\mu_{\backslash k}^*$, the mean of the tilted distribution itself.) However, once again, I'm unsure if we could use sample covariances of the simulations to approximate $\Sigma_{\backslash k}$.\\

\noindent (c) When I programmed EP for the bioassay example, after a few iterations I was getting non-positive-definite inverse matrices for both steps. I'm guessing this will end up being an issue here as well.

%CODE
%\pagebreak
%\section{Code}
%\texttt{\lstinputlisting{hierarchical_model.R}}

\end{document}